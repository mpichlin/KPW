\documentclass[a4paper,11pt,polish]{memoir}
% acelnosxz
% ����󶼿
% ��ʣ�Ӧ��
\usepackage[cp1250]{inputenc}
\usepackage[T1]{fontenc}
\usepackage{pslatex}
\usepackage[polish]{babel}
\usepackage{tabularx}
\usepackage{multicol}
\usepackage{color,colortbl}
\usepackage{calc,soul,fourier}
\usepackage{setspace}
\usepackage{indentfirst}
\usepackage{type1cm,eso-pic}
\usepackage{ifpdf}
\usepackage{amsmath}  
\usepackage{listings}
%\usepackage[nottoc]{tocbibind}
%\usepackage[titles]{tocloft}
%\usepackage{natbib}
\usepackage[sectionbib]{chapterbib}
\makeatletter
\renewcommand{\@memb@bsec}{\section*{\bibname}\prebibhook}
\makeatother

\usepackage{makeidx} \makeindex 
\ifpdf
 \usepackage[pdftex,bookmarks,breaklinks]{hyperref}
 \usepackage[pdftex]{graphicx}
 \DeclareGraphicsExtensions{.pdf,.jpg,.mps,.png}
 \pdfcompresslevel=9
\else
 \usepackage{graphicx}
 \DeclareGraphicsExtensions{.eps,.ps,.jpg,.mps,.png}
\fi

\makeatletter
\AddToShipoutPicture{%
  \setlength{\unitlength}{1mm}
  \put(18,29.5){\line(-1,0){10}}%
  \put(21,26.5){\line(0,-1){10}}%
  \put(192,29.5){\line(1,0){10}}%
  \put(189,26.5){\line(0,-1){10}}%
  \put(18,267.5){\line(-1,0){10}}%
  \put(21,270.5){\line(0,1){10}}%
  \put(192,267.5){\line(1,0){10}}%
  \put(189,270.5){\line(0,1){10}}%
}
\makeatother

%%%%%%%%%%%%%%%%%%%%%%%%%%%%%%%%%%%%%%%%%%%%%%%%%%%%%%%%%%%%%%%%%%%%%%%%%%%%%%%%%%%%%%%%%%%%%%%%%%%%%
%%    Ustawienia strony + definicje pomocnicze                                                     %%
%%%%%%%%%%%%%%%%%%%%%%%%%%%%%%%%%%%%%%%%%%%%%%%%%%%%%%%%%%%%%%%%%%%%%%%%%%%%%%%%%%%%%%%%%%%%%%%%%%%%%
%\setlength{\headsep}{0pt} \setlength{\headheight}{0pt}
\setlength{\hoffset}{0mm} %%a4
\setlength{\voffset}{0mm} %-14mm %%a4
\setlength{\footskip}{23pt} % 23pt ~= 8mm
\setlength{\topmargin}{11mm}%{7.65mm} %28pt -2.5mm
\setlength{\oddsidemargin}{11.85mm}
\setlength{\evensidemargin}{11.85mm}

\setlength{\textwidth}{135.5mm}
\setlength{\textheight}{203.6mm}

\setlength{\parindent}{18pt}
\setlength{\parskip}{0pt} %{1ex plus 0.5ex minus 0.2ex}

\setlength{\extrarowheight}{2pt}

\sloppy  % wyr�wnanie z dw�ch stron
\renewcommand{\topfraction}{1.0}
\renewcommand{\bottomfraction}{1.0}
\renewcommand{\textfraction}{0.0}

%\widowpenalty=10000 % ostatni wiersz akapitu nie zostanie przeniesiony na nast�pn� stron� 
%\clubpenalty=10000 % pierwszy wiersz akapitu nie b�dzie ko�czy� strony (nie u�ywam tego ustawienia)
%\tolerance = 500 \pretolerance = 900 %% sk�ad z wi�ksz� `tolerancj�' (mo�na te warto�ci zwi�kszy� bardziej)
%\hbadness= 1450 %% zmniejsza licz� wy�wietlanych ostrze�e� (mo�na zwi�kszy�, ale bez przesady)
%\hfuzz = 1.5pt %% tekst mo�e stercze� ma marginesie na 1,5pt (ok. 0,5mm)

%\setlength{\floatsep}{12pt}
%\setlength{\textfloatsep}{12pt}
%\setlength{\intextsep}{12pt}
%\setlength{\arrayrulewidth}{0.5pt}

\makeatletter
\renewenvironment{itemize}{
  \begin{list}{  
  \csname labelitem\romannumeral\the\@listdepth\endcsname}{
    \setlength{\leftmargin}{1em}
	\setlength{\topsep}{6pt}%
	\setlength{\partopsep}{0pt}%
	\setlength{\parskip}{0pt}%
	\setlength{\parsep}{0pt}%
	\setlength{\itemsep}{0pt}}
}{
  \end{list}
}
\renewenvironment{quote}{
  \begin{list}{}{}
	 \item[]}
	 {\unskip\end{list}}
\makeatother

\makeatletter
\newskip\@hlskip
%\@hlskip=.5\baselineskip \@plus 1mm \@minus .5mm
\@hlskip=6pt

\newdimen\verbatimleftmargin
  \verbatimleftmargin\z@
\newdimen\verbatimbaselineskip
  \verbatimbaselineskip\baselineskip
\def\verbatimsize{\normalsize}

\def\@verbatim{%
 \topsep\@hlskip
 \partopsep\z@\parsep\z@\itemsep\z@
 \trivlist \item\relax
  \if@minipage\else
   \vskip\baselineskip
   \vskip-\verbatimbaselineskip
%  \vskip\parskip
  \fi
  \leftskip\@totalleftmargin
  \if@minipage\else
   \advance \leftskip by \verbatimleftmargin
  \fi
  \rightskip\z@skip
  \parindent\z@\parfillskip\@flushglue\parskip\z@skip
  \@@par
  \@tempswafalse
  \def\par{%
    \if@tempswa
      \leavevmode \null \@@par\penalty\interlinepenalty
    \else
      \@tempswatrue
      \ifhmode\@@par\penalty\interlinepenalty\fi
    \fi}%
  \let\do\@makeother \dospecials
  \obeylines 
   \verbatimsize \baselineskip\verbatimbaselineskip
   \verbatim@font \@noligs
  \everypar \expandafter{\the\everypar \unpenalty}%
}

\makeatother

\definecolor{nicered}{rgb}{.647,.129,.149}
\newcommand{\autorzy}{}

\makeatletter
\newlength\dlf@normtxtw
\setlength\dlf@normtxtw{\textwidth}
\def\myhelvetfont{\def\sfdefault{mdput}}
\newsavebox{\feline@chapter}
\newcommand\feline@chapter@marker[1][4cm]{%
\sbox\feline@chapter{%
\resizebox{!}{#1}{\fboxsep=1pt%
\colorbox{nicered}{\color{white}\bfseries\sffamily\thechapter}%
}}%
\rotatebox{90}{%
\resizebox{%
\heightof{\usebox{\feline@chapter}}+\depthof{\usebox{\feline@chapter}}}%
{!}{\scshape\so\@chapapp}}\quad%
\raisebox{\depthof{\usebox{\feline@chapter}}}{\usebox{\feline@chapter}}%
}
\newcommand\feline@chm[1][4cm]{%
\sbox\feline@chapter{\feline@chapter@marker[#1]}%
\makebox[0pt][l]{% aka \rlap
\makebox[1cm][r]{\usebox\feline@chapter}%
}}
\makechapterstyle{daleif1}{
\renewcommand\chapnamefont{\normalfont\Large\scshape\raggedleft\so}
\renewcommand\chaptitlefont{\normalfont\huge\bfseries\scshape\color{nicered}}
\renewcommand\chapternamenum{}
\renewcommand\printchaptername{}
\renewcommand\printchapternum{\null\hfill\feline@chm[2.5cm]\par}
\renewcommand\afterchapternum{\par\vskip\midchapskip}
\renewcommand\printchaptertitle[1]{\chaptitlefont\raggedleft ##1\\  \makebox[\textwidth - (1cm + \widthof{\Huge \thechapter.})][r]{\Large \itshape \autorzy}\par}
}

%definicja nag��wk�w
\renewcommand{\chaptermark}[1]{\markboth{\ifnum \c@secnumdepth >\m@ne
      \thechapter \ \fi #1}{}} 
\renewcommand{\sectionmark}[1]{\markright{\thesection \ #1}{}} 
\setcounter{secnumdepth}{3}
%definicja g��boko�ci numerowania sekcji
\maxsecnumdepth{subsection}

% z jakiego� powodu czcionki s� mniejsze o jeden punkt ni� wynika�oby to z poni�szych ustawie�
\renewcommand{\bfdefault}{b}
\renewcommand{\normalsize}{\fontsize{11pt}{12.5pt}\selectfont}
\newcommand{\smallp}{\fontsize{9.5pt}{11.5pt}\selectfont}
\makeatother

%%% Strona tytu�owa
\newcommand*{\titleTH}{\begingroup% T&H Typography
\raggedleft
\vspace*{4cm}
{\bfseries Kolekcja prac}\\[\baselineskip]
{\textcolor{nicered}{\Huge 	KOMPUTEROWE}}\\[\baselineskip]
{\textcolor{nicered}{\Huge 	PRZETWARZANIE}}\\[\baselineskip]
{\textcolor{nicered}{\Huge 	WIEDZY}}\\[\baselineskip]
\vfill
{\Large Politechnika Wroc�awska, 2011}\par
\vspace*{3\baselineskip}
\endgroup}

\begin{document}
\thispagestyle{empty} \titleTH \cleardoublepage
\tableofcontents*
\thispagestyle{empty} \cleardoublepage

\chapterstyle{daleif1}
\bibliographystyle{myaplike}
\renewcommand{\autorzy}{P. Batog, M. Pichli�ski}
\chapter[Budowa modeli informacyjnych i ich profilowanie]
        [Budowa modeli informacyjnych i ich profilowanie]
        {Budowa modeli informacyjnych i ich profilowanie}

\section{Wst�p}
Rozw�j komputer�w, sprz�tu elektronicznego oraz Internetu sprzyja zwi�kszaniu
ilo�ci informacji oraz daje coraz to �atwiejszy do niej dost�p. W roku 2003
za pomoc� wyszukiwarek internetowych (tzw. powierzchniowy Internet)
by�o dost�pne 167 terabajt�w danych (w por�wnaniu do 2000 roku jest ich
trzykrotnie wi�cej) \cite{HowMuchInformation}.
IBM w 2006 roku stwierdzi�o, �e w 2010 ilo�� przechowywanej elektronicznej
informacji b�dzie zwi�ksza�a si� dwukrotnie w przeci�gu 11 godzin 
(nie koniecznie w samej sieci) \cite{ToxicTerabyte}.

Dodatkowo ,,powierzchnia'' Internetu to wierzcho�ek g�ry lodowej, poniewa�
ukryty (g��boki) Internet, czyli taki kt�rego dane nie s� dost�pne 
bezpo�rednio przez hiperlinki (np. wewn�trzne bazy danych) szacuje si� na 500
krotnie wi�kszy ni� to co na codzie� mo�na otrzyma� dzi�ki Google 
\cite{DeepWeb}. Du�a ilo�� informacji powoduje: utrudnienie uzyskania
potrzebnych danych, wzrost chaosu informacyjnego, spowolnienie dost�pu do
zasob�w z powodu obci��enia powsta�ego w wyniku przeszukiwania
\cite{ToxicTerabyte}. Problemy wynikaj�ce z konieczno�ci przetwarzania zasob�w 
wiedzy, pr�buje si� rozwi�za� za pomoc� modeli danych strukturalnych. 

W niniejszej pracy zostanie przedstawione ontologiczne podej�cie w zagadnieniu
budowy modelu informacji, na przyk�adzie tworzenia tezaurusa. W kolejnych
sekcjach tego rozdzia�u om�wione s� poj�cia modelu informacyjnego oraz 
ontologii. W cz�ci czwartej przedstawiono czym jest tezaurus, jaki jest
proces jego tworzenia oraz jakie ma zastosowania. Natomiast w sekcji ostatniej
opisany jest j�zyk SKOS, w kt�rym zosta�a zrealizowana cz�� praktyczna
projektu.

\section{Czym s� modele informacyjne?}
Modele informacyjne s� to sposoby reprezentacji zasob�w wiedzy, takich jak:
koncepty, relacje, ograniczenia, zasady, operacje. Nadaj� one danym znaczenia 
semantycznego \cite{Tina}. Dzi�ki modelom informacyjnym, wiedza, kt�ra jest
w nich przechowywana, nabiera elastyczno�ci pozwalaj�cej na dzielenie jej
pomi�dzy r�nymi aplikacjami, �atwiejsze wyszukiwanie po��danych informacji 
oraz dostosowanie do potrzeb u�ytkownika \cite{GilbaneReport}.


Warto wspomnie� o r�nicy pomi�dzy modelami informacyjnymi a modelami danych,
kt�re mog� by� mylnie uto�samiane ze sob�. Celem tych pierwszych jest 
zarz�dzanie modelami na poziomie ideowym oraz przedstawienie relacji pomi�dzy
nimi, niezale�nie od konkretnych implementacji b�d� protoko��w transmisji
danych. Model informacyjny jest narz�dziem przeznaczonym dla projektant�w.
Natomiast modele danych odnosz� si� ni�szemu poziomowi abstrakcji, zawieraj�
wi�cej szczeg��w, s� one ju� konkretn� implementacj� modeli informacyjnych 
\cite{RFC3444}.



\section{Ontologia jako model informacji}
Celem stworzenia modelu informacji jest nadanie jej wy�szego stopienia organizacji. Uporz�dkowanie ma za� umo�liwi� �atwiejsze przetwarzanie informacji i wyci�gania na ich podstawie okre�lonych wniosk�w. Innymi s�owy jest to dodanie abstrakcji umo�liwiaj�cej przechodzenie od informacji w kierunku do wiedzy. Budowa modelu informacji jest zagadnieniem z natury heurystycznym. Nie istniej� bowiem og�lne prawa, wed�ug kt�rych proces budowy takiego modelu mia�by zosta� przeprowadzony. 

Budowa modelu �ci�le wi��e si� z dziedzin� i problemem, kt�rego ma by� rozwi�zaniem. W dzisiejszych czasach dost�pnych jest mn�stwo odpowiednich metod i technik. Co wi�cej poszczeg�lne organizacje modyfikuj� og�lnie znane metody bazuj�c na do�wiadczeniach w konkretnych zagadnieniach i technikach. Taka sytuacje bywa nazywana w literaturze ,,d�ungl� metodologiczn�'' \cite{Verhoef}. W niniejszej pracy skupiono si� na podej�ciu ontologicznym.

\subsection{Ontologia}
Ontologia, jest reprezentacj� dystrybuowanej konceptualizacji okre�lonej domeny\footnote{Hodge}. M�wi�c pro�ciej, jest budow� wsp�lnej warstwy poj�ciowej, rozwini�ciem konceptualizacji poprzez dodanie relacji pomi�dzy poszczeg�lnymi podmiotami. Obydwa poj�cia maj� swoje korzenie w naukach filozoficznych.
\begin{itemize}
   \item Konceptualizacja - pr�ba okre�lenia �cis�ych poj��, (usuni�cie wieloznaczno�ci) niezb�dnych do opisu jednoznacznie rozumianego procesu z okre�lonej dziedziny.
	\item Ontologia - (od greckich $o \nu \tau o$ - byt i $\iota o\gamma o\varsigma$ - nauka ) najog�lniejsza nauka rozwa�aj�ca poj�cie bytu, do XVII w uto�samiana z metafizyk�. Jej pocz�tk�w mo�na szuka� u Arystotelesa. Dziedzina metafizyki, zwi�zana z badaniem, wyja�nianiem natury, kluczowych w�a�ciwo�ci oraz relacji rz�dz�cych wszelakimi bytami b�d� g��wnych zasad i przyczyn bytu \footnote{Webster}.
\end{itemize}  

W praktyce technologii informacyjnej poprzez ontologi� rozumie si� jako  �wiadom�, formaln� 
specyfikacj� koncept�w (poj��) w danej dziedzinie i relacji pomi�dzy nimi. Ontologia ��czy ze sob� nie tylko wyra�one w niej poj�cia, ale r�wnie� wiedz�, kt�ra mo�e by� na jej podstawie wywnioskowana. Jako spos�b reprezentacji zintegrowanego  modelu wiedzy dziedziny oraz jako instrument badawczy tre�ci informacji tekstowej razem z innymi technologiami informatycznymi jest obecnie jednym z kluczowych element�w 
rozwijaj�cej si� idei spo�ecze�stwa informacyjnego. Dowodem na to jest dynamiczny rozw�j internetu trzeciej generacji \footnote{Semantic Web, Web 3.0 - http://www.slideshare.net/skruk/sie-semantyczna-w-teorii-i-praktyce}. 
 
\section{Tezaurus}
Tezaurus w og�lnym rozumieniu stanowi s�ownik termin�w bliskoznacznych. S�owo pochodzi z greckiego $\Theta \eta \sigma \alpha \upsilon \rho o \varsigma$, gdzie oznacza�o zbi�r rzeczy o wielkiej warto�ci (magazyn, skarbiec). W technologii informacyjnej i bibliotekoznawstwie tezaurus jest �ci�le okre�lonym\footnote{przyk�adowe normy s� prezentowane w kolejnym rozdziale}  rozwini�ciem s�ownika. Jest to zbi�r semantycznie i hierarchicznie powi�zanych termin�w, u�atwiaj�cy wyszukiwanie informacji. 

Tezaurus jest pewnego rodzaju zastosowaniem podej�cia ontologicznego w celu sklasyfikowania s��w j�zyka naturalnego. Jak wiadomo, j�zyk naturalny zawiera wiele s��w o wielu znaczeniach, ale relacje pomi�dzy s�owem a jego znaczeniem nie s� jednoznaczne. Jedno s�owo bowiem mo�e mie� wiele znacze�, a jedno poj�cie mo�e by� wyra�ane przez wiele s��w. Co wi�cej cz�sto relacje te zmieniaj� si� w zale�no�ci od kontekstu jak i samy os�b je wypowiadaj�cych i czytaj�cych. Cz�owieka interesuje przewa�nie znaczenie\footnote{oczywi�cie, je�li pominie si� aspekty natury estetycznej}, jednak musi si� on pos�ugiwa� si� s�owami. 

Idea�em tezaurusa jest system komputerowy, kt�ry po wprowadzeniu pewnych s��w, odpowiednio je zinterpretuje i zdefiniuje koncepty poszukiwane przez u�ytkownika, dzi�ki czemu mo�liwe b�dzie zwr�cenie dok�adnie wszystkich wynik�w interesuj�cych u�ytkownika, ale bez �adnego nadmiaru. Przyk�adowo, na zapytanie ,,lod�wka'' system wy�wietli tak�e to co by�o opisane jako ,,ch�odziarko-zamra�arka'', ale ju� nie jako ,,ch�odnie''. 

Istota inteligentna bez wi�kszego trudu uczy si� relacji zachodz�cych pomi�dzy s�owami i znaczeniami, cho� proces ten trwa przez ca�e �ycie. Jednak nawet dla ludzi jest to nietrywialne zagadnienie, o czym �wiadcz� tak zwane nieporozumienia, wyst�puj�ce gdy odbiorca zrozumia� zupe�nie co innego, ni� nadawca mia� na my�li. �wiadczy o tym tak�e szacunek dla ludzi, kt�rzy opanowali umiej�tno�� jasnego i zwi�z�ego wypowiadania si�. Jak jednak nauczy� tego maszyn�, by umo�liwi� komputerowe przetwarzanie wiedzy i wyszukiwanie informacji w ogromnych zbiorach danych?

\subsection{Katalog przedmiotowy}
Pierwsz� pr�b� semantycznego usystematyzowania termin�w podj�li ju� dawno bibliotekarze. By�o ni� utworzenie katalogu przedmiotowego. W takim systemie ka�da z ksi��ek opisana jest jednym lub wieloma has�ami z katalogu. Zak�ada si�, �e opisywanie zawarto�ci tematycznej ksi��ek, jak i wyszukiwanie odbywa si� przy u�yciu tego samego zbioru s��w. Niestety katalog taki jest ma�o elastyczny i odporny a efektywne korzystanie z niego jest rzecz� wymagaj�c� odpowiedniego przygotowania. O ile do�wiadczony bibliotekarz na co dzie� obcuj�cy z katalogiem nie ma problem�w z okre�leniem odpowiednich kategorii, to przypadku przeci�tnego u�ytkownika cz�� zasob�w pozostanie ukryta, bowiem jego intuicje co do tego do jakiej kategorii zosta� zaklasyfikowany poszukiwany przez niego temat mog� by� mylne. Co wi�cej odwo�uj�c si� do has�a np. \textit{Rzeki w Polsce} mo�na znale�� jedynie pozycje omawiaj�ce og�lnie Rzeki, natomiast ju� nie odnajdzie si� monografii o Odrze, Wi�le itd. ani ksi��ek opisuj�cych og�lnie rzeki w Europie. U�ytkownik musia�by odnale�� w katalogu poj�cia w�sze i szersze a nast�pnie przeszuka� listy ksi��ek im odpowiadaj�ce, co z pewno�ci� by�oby czasoch�onne.  

Komputeryzacja katalogu zwi�ksza wygod� u�ytkowania i efektywno�� systemu, m.in. poprzez znaczne zwi�kszenie ilo�ci kategorii mo�liwych do przeszukania w tym samym czasie, jednak nie zmienia zasady dzia�ania, i nie usuwa jego zasadniczych wad. Co wi�cej z komputerowych katalog�w korzystaj� z regu�y bez pomocy bibliotekarzy sami czytelnicy, przewa�nie nie u�wiadamiaj�cy sobie opisanych wad katalog�w, co mo�e zniech�ca� do odwiedzania bibliotek i prowadzi� do frustracji.
\subsection{S�owa kluczowe}
W ostatnich latach do opisywania danych coraz cz�ciej pos�ugiwano si� technologi� s��w kluczowych. Z uwagi na ich ,,lu�ny'' charakter nale�y stwierdzi�, �e jest to technologia komplementarna w stosunku do s�ownictwa kontrolowanego \cite{KasperekKonf}. Oczywi�cie bardzo u�yteczna, posiada jednak szereg wad takich jak:
\begin{itemize}
   \item konieczno�� podawania synonim�w
	\item niepe�ny opis
	\item brak usystematyzowania
	\item zwracanie setek bezu�ytecznych trafie�\footnote{por. wyszukiwarki internetowe}
	\item brak relacji
	\item trudno�ci ze zbudowaniem systemu inteligentnego
\end{itemize}    

\subsection{Wyszukiwanie pe�notekstowe}
Zamiast ogranicza� si� do samego opisu danego podmiotu, mo�na rozwa�y� jako alternatyw� odwo�anie si� do pe�nego tekstu - ksi��ki, artyku�u, itp. Niestety nie wszystko posiada sw�j ,,pe�ny tekst''. Obiektem kategoryzacji i wyszukiwania s� nie tylko ksi��ki i artyku�y, ale tak�e dzie�a sztuki, utwory muzyczne, artyku�y sklepowe i wiele innych. Warto zauwa�y�, �e sama zawarto�� bibliotek jest rzadko w pe�ni zdigitalizowana a pe�ny tekst dost�pny w formie elektronicznej. Ponadto takie rozwi�zanie tak�e ma swoje ograniczenia, m.in.
 \begin{itemize}
   \item znacznie d�u�szy czas trwania
	\item zwracanie znacznie wi�kszej ilo�ci bezu�ytecznych trafie�
	\item ograniczenie do jednego j�zyka naturalnego.
\end{itemize}

\subsection{Proces tworzenia tezaurusa}
Tezaurus w uproszczeniu mo�na rozumie� jako zbi�r s�ownictwa kontrolowanego i zbi�r relacji zachodz�cych pomi�dzy s�owami.
S�ownictwo kontrolowane jest ujednoznacznione (poprzez jawne zdefiniowanie poj�� lub rozr�nienie pomi�dzy r�nymi znaczeniami tego samego s�owa). Przewa�nie zbi�r s��w jest ograniczony do jakiej� konkretnej dziedziny, w kt�rej b�dzie wykorzystywany. Do opisu zasob�w wolno pos�ugiwa� si� tylko s�owami kt�re s� w systemie. 

Nast�pnym krokiem w organizacji wiedzy jest okre�lenie regu� relacyjnych zachodz�cych pomi�dzy danymi konceptami. Koncept jest poj�ciem abstrakcyjnym, mo�e odpowiada� mu wiele s��w, ale mo�e te� nie istnie� w�a�ciwe s�owo. Najcz�ciej spotyka si� implementacj� nast�puj�cych regu�:
\begin{itemize}
   \item Hiperonimia - terminy nadrz�dne (koncepty og�lniejsze)
	\item Hiponimia - terminy podrz�dne (koncepty bardziej szczeg�owe)
	\item Synonimia  - termin bliskoznaczny (wymienienie Deskryptor�w - termin�w preferowanych i Askryptor�w)
	\item Homonimia - r�ne znaczenie tego samego s�owa (stworzenie odr�bnych koncept�w)
	\item Asocjacja - skojarzenie ze sob� pokrewnych koncept�w
\end{itemize} 
Poprzez dodanie relacji hierarchicznych i skojarzeniowych, otrzymuje si� narz�dzie, kt�re umo�liwia stworzenie inteligentnego systemu wyszukiwawczego oferuj�cego znacznie wi�ksze mo�liwo�ci ni� zwyk�y katalog.   
Jadwiga Wo�niak-Kasperek \cite{KasperekPoradnik} wymienia nast�puj�ce etapy budowy tezaurusa: 
\begin{enumerate}
   \item Okre�lenie zakresu j�zyka
	\item Wskazanie  �r�de�, z kt�rych (lub na podstawie kt�rych) b�dzie pobierane s�ownictwo
	\item Robocze okre�lenie struktury cz�ci rzeczowej (systematycznej) tezaurusa 
	\item Wyb�r metody gromadzenia s�ownictwa
	\item Zgromadzenie s�ownictwa
	\item Ustalenie zasad tworzenia ci�g�w synonimicznych
	\item Ustalenie, jak b�d� ,,oddzielane'' r�ne znaczenia wyra�e� wieloznacznych
	\item Wybranie sposobu zapisu deskryptor�w, askryptor�w, ew. modyfikator�w, przyj�cie notacji, za pomoc� kt�rej b�d� oznaczane 
typy relacji ��cz�cych has�a w tezaurusie.
	\item  Opracowanie kilku wzorcowych przyk�ad�w artyku��w deskryptorowych i 
askryptorowych (z komentarzem)
	\item Opracowanie tezaurusa, w tym cz�ci systematycznej
\end{enumerate}
\subsection{Przyk�ady tezaurus�w}
Od XVI w nazwa tezaurus wyst�puje jako oznaczenie s�ownika lub leksykonu. W Polsce, pierwszym dzie�em tego typu by� \textit{ Thesaurus Polono-Latino-Graecus} Grzegorza Knapiusza (Krak�w 1621-32). Tw�rc� pierwszego nowo�ytnego, usystematyzowanego semantycznie tezaurusa by� Peter Mark Roget. \textit{Thesaurus of English Words and Phrases}, dzie�o jego �ycia ukaza�o si� w 1852r w Londynie i jest u�ywane tak�e dzi�, cho� oczywi�cie nieustannie aktualizowane. Tezaurus Rogeta dotyczy� j�zyka angielskiego i zawiera� 6 poziom�w hierarchicznych. Tezaurusy jako j�zyk wyszukiwawczo-informacyjny zacz�y si� intensywnie rozwija� w drugiej po�owie XX w. i trend ten trwa w�a�ciwie do dzisiaj. Pierwszy polski tezaurus ukaza� si� w 1969 r. i by� wykazem termin�w z zakresu urz�dze� budowlanych i transportu bliskiego \cite{Sosinska}. Budow� (tak�e wieloj�zycznych) tezaurus�w m.in. w zakresie dziedzictwa kulturowego zajmuje si� wiele o�rodk�w naukowych w Europie (korzystaj�c tak�e ze wsparcia Unii Europejskiej).
Przyk�adowe tezaurusy dost�pne on-line:
\begin{itemize}
    \item Roget Thesaurus http://www.roget.org/
	\item WordNet http://wordnetweb.princeton.edu/perl/webwn
	\item Visual Thesaurus http://www.visualthesaurus.com/
	\item Tezaurus Dziedzictwa Kulturowego 

http://historiasztuki.uni.wroc.pl/tezaurus.html
\end{itemize}  

\subsection{Wykorzystanie tezaurus�w}
Tezaurusy mog� by� stosowane wsz�dzie tam gdzie zachodzi potrzeba wyszukiwania lub ,,zrozumienia'' tekstu przez komputer. Przyk�adowo b�d� to: 
\begin{itemize}
   \item Biblioteki
	\item Archiwa
	\item Muzea
	\item Bazy danych
	\item Wyszukiwarki
	\item Rozbudowane sklepy internetowe i serwisy aukcyjne
	\item Systemy przetwarzania i rozumienia mowy
\end{itemize}
\section{Normy}
Za wzorzec tezaurusa uwa�a si� opublikowany w 1967 r. Thesaurus of Engineering and Scientific Terms (TEST), kt�ry powsta� w ramach tzw. projektu LEX, realizowanego w latach sze��dziesi�tych wsp�lnie przez przez Office of Naval Research ameryka�skiego Department of Defence (DOD) oraz Engineering Joint Council (EJC). Mia� on form� tzw. alfabetyczno-hierarchicznego wykazu deskryptor�w i askryptor�w, w kt�rym zastosowano symetryczne oznaczenia relacji ekwiwalencji (USE � UF), hierarchicznych (BT i NT) oraz asocjacyjnych (RT) u�ywanych do dzisiaj \cite{Sosinska}. Na podstawie w�a�nie tego systemu powsta�a w USA pierwsza na �wiecie norma dotycz�ca budowy tezaurus�w - ANSI Z39.19-1974.
Pierwszym mi�dzynarodowym standardem by�a norma ISO 2788 \textit{Documentation � Guidelines for the Establishment and Development of Monolingual Thesauri}, przyj�ta w Polsce jako PN/N-09008. Z pocz�tkiem XXI w. w zwi�zku z konieczno�ci� zapewnienia odpowiedniej elastyczno�ci (interoperacyjno�ci pomi�dzy tezaurusami i innymi systemami organizacji wiedzy) wydane zosta�y nowe normy, ameryka�ska Z39.19-2005 \textit{Guidelines for Construction, Format, and Management of Controlled Vocabularies} (ANSI-NISO, 2005), brytyjska BS 8723 \textit{Structured Vocabularies for Information Retrieval}(BSI, 2005a; BSI, 2005b), a w trakcie budowy jest mi�dzynarodowa norma ISO 25964\footnote{http://www.niso.org/workrooms/iso25964/}.
Diagram klas wchodz�cych w sk�ad tezaurusa, wed�ug normy BS8723\footnote{http://schemas.bs8723.org/} zosta� umieszczony na rysunku 1. 

\begin{figure}[ht]
\centering
\includegraphics[width=13cm]{./Model.pdf}
\caption{Klasy wchodz�ce w sk�ad tezaurusa, wed�ug normy BS8723}
\end{figure}


\section{Specyfikacje techniczne - j�zyk SKOS}
\emph{Simple Knowledge Organization System} jest j�zykiem, kt�ry pozwala na
tworzenie system�w organizacji wiedzy jak tezaurusy, schematy klasyfikacji,
systemy nag��wkowe oraz taksonomie. Modele, kt�re s� zapisane za pomoc� SKOS,
mog� by� �atwo przetwarzane przez maszyny, wymieniane pomi�dzy aplikacjami
komputerowymi oraz publikowane w internecie.

Modele informacyjne utworzone w j�zyku SKOS spe�niaj� warunki 
\emph{OWL Full ontology} i mog� one zawiera� r�wnie� informacje zapisane za
pomoc� OWL (\emph{Web Ontology Language}). SKOS podobnie jak OWL powsta� w
oparciu o twierdzenia RDF (\emph{Resource Description Framework}) i formalna 
sk�adnia jak� si� pos�uguje to RDF/XML oraz Turtle \cite{SKOS-ref}. 

\subsubsection{S�ownictwo}
Zasoby s�ownictwa jakim pos�uguje si� SKOS znajduj� si� pod adresem:
\url{http://www.w3.org/2004/02/skos/core#}, w niniejszym dokumencie b�dzie on
opisany skr�tem: \textbf{skos:}. Na podstawie \cite{SKOS-pr} dost�pne s�
nast�puj�ce s�owa kluczowe, kt�re definiuj� klasy (s�owa zaczynaj�ce si� z
du�ej litery) oraz w�a�ciwo�ci (s� pisane z ma�ej litery):
\begin{itemize}
  \item \textbf{skos:Concept} - jednostka my�li, b�d�ca abstrakcyjn� encj�
    niezale�n� od poj�� u�ytych do zetykietowania jej; semantyczna zawarto��
    danego konceptu, mo�e zosta� wyra�ona za pomoc� kombinacji innych koncept�w
    \cite{Willpower}:
  \item Etykiety - s� cechami koncept�w, kt�re s�u�� jako odno�niki do nich
    wyra�one w j�zyku naturalnym. Skos pozwala na tworzynie etykiet w r�nych
    j�zykach w postaci:
\begin{verbatim}
"tre�� etykiety"@skr�t_j�zyka
\end{verbatim}
    Istniej� trzy rodzaje etykiet, kt�re wzajemnie si� wykluczaj�, wi�c nie
    dozwolone jest istnienie etykiet innego typu o tej samej tre�ci
    \begin{itemize}
      \item \textbf{skos:prefLabel} - preferowana etykieta, mo�e by� tylko
        jedna dla danego konceptu; nie jest zabronione by r�ne koncepty
        zawiera�y t� sam� etykiet�, jednak zaleca si� by takich przypadk�w
        nie by�o
      \item \textbf{skos:altLabel} - alternatywna etykieta, przydatna do
        okre�lenia synonim�w, jak r�wnie� wyraz�w bliskoznacznych oraz
        akronim�w
      \item \textbf{skos:hiddenLabel} - ukryta etykieta, dost�pnie jedynie
        wewn�trznie przez aplikacj�, w celu indeksowania i wyszukiwania, nie
        jest ona widoczna dla u�ytkownika; mo�e s�u�y� do uwzgl�dniania
        liter�wek podczas poszukiwania danego konceptu
    \end{itemize}
  \item Relacje - s�u�� do linkowania koncept�w pomi�dzy sob�, SKOS dostarcza
    nast�puj�ce typy relacji:
    \begin{itemize}
      \item \textbf{skos:narrower} - hierarchiczne po��czenie z konceptem
        bardziej szczeg�owym na przyk�ad:
\begin{verbatim}
ex:figuraGeometryczna rdf:type skos:Concept;
  skos:narrower ex:kwadrat.
ex:kwadrat rdf:type skos:Concept.
\end{verbatim}
      \item \textbf{skos:broader} - ��czy z konceptem bardziej og�lnym:
\begin{verbatim}
ex:kwadrat rdf:type skos:Concept;
  skos:broader ex:figuraGeometryczna.
ex:figuraGeometryczna rdf:type skos:Concept.
\end{verbatim}
      \item \textbf{skos:related} - ��czy z innym konceptem bez
        ustalania hierarchii, nie jest relacj� tranzytywn�
    \end{itemize}
    Poj�cia skos:broader i skos:narrower s� przeciwie�stwami, wi�c wystarczy
    okre�li� tylko jedn� z tych relacji, druga natomiast zostanie automatycznie
    przypisania w procesie wnioskowania OWL. Dodatkow� cech� wy�ej wymienionych
    relacji jest brak ich tranzytywno�ci, czyli:
\begin{verbatim}
ex:figuraGeometryczna rdf:type skos:Concept;
  skos:narrower ex:prostokat.
ex:prostokat rdf:type skos:Concept;
  skos:narrower ex:kwadrat.
ex:kwadrat rdf:type skos:Concept.
\end{verbatim}
    nie implikuje relacji:
\begin{verbatim}
ex:figuraGeometryczna skos:narrower ex:kwadrat.
\end{verbatim}
    Jednak w \cite{SKOS-ref} opisane s� tranzytywne wersje relacji
    \emph{broader} i \emph{narrower}: \textbf{skos:broaderTransitive},
    \textbf{skos:narrowerTransitive}, kt�re umo�liwiaj� z powy�szej deklaracji
    wnioskowa�:
\begin{verbatim}
ex:figuraGeometryczna skos:narrowerTransitive ex:kwadrat.
\end{verbatim}
    Nale�y pami�ta� r�wnie� o tym, �e poszczeg�lne typy relacji wzajemnie si�
    wykluczaj�.
  \item Przypisy dokumentacyjne - informacje w postaci zrozumia�ej dla
    cz�owieka, przypisane do konceptu, w celu jego opisania:
    \begin{itemize}
      \item \textbf{skos:definition} - zawiera pe�ne wyja�nienie znaczenia 
        danego konceptu
      \item \textbf{skos:example} - zawiera przyk�ady u�ycia konceptu
      \item \textbf{skos:scopeNote} - dostarcza cz�ciowego znaczenia konceptu
        w szczeg�lno�ci okre�la zakres jego u�ycia
      \item \textbf{skos:historyNote} - opisuje znacz�ce zmiany jakie zasz�y w 
        znaczeniu, b�d� formie konceptu
      \item \textbf{skos:editorialNote} - informacja od autora, umieszczana 
        w celu u�atwienia utrzymania konceptu
      \item \textbf{skos:changeNote} - informacja o zmianach w koncepcie,
        umieszczona w celach administracyjnych
    \end{itemize}
  \item Schematy koncept�w - s�u�� do grupowania, pojedynczych koncept�w, w
    celu u�atwienia indeksowania i tworzenia np. tezaurus�w:
    \begin{itemize}
      \item \textbf{skos:ConceptScheme} - kontener b�d�cy schematem koncept�w,
        zawieraj�cym inne koncepty
      \item \textbf{skos:hasTopConcept} - przypisuje danemu schematowi,
        najwy�szy w hierarchii koncept; mo�e by� kilka najwy�szych koncept�w
      \item \textbf{skos:topConceptOf} - przypisuje danemu konceptowi
        schemat koncept�w, w kt�rym b�dzie on najwy�szy w hierarchii
      \item \textbf{skos:inScheme} - przypisuje danemu konceptowi schemat
        koncept�w, w kt�rym si� znajduje, bez okre�lenia hierarchii
    \end{itemize}
    Niestety w \cite{SKOS-ref} okre�lone jest, �e relacje pomi�dzy konceptami
    (tzn. broader, narrower, itp.) nie s� przenoszone na schemat koncept�w,
    czyli:
\begin{verbatim}
ex:tezaurusAstronomiczny rdf:type skos:ConceptScheme;
  hasTopConcept ex:cialoNiebieskie.
ex:cialoNiebieskie rdf:type skos:Concept;
  skos:narrower ex:gwiazda.
ex:gwiazda rdf:type skos:Concept.
\end{verbatim}
    nie implikuje:
\begin{verbatim}
ex:gwiazda skos:inScheme ex:tezaurusAstronomiczny
\end{verbatim}
    \item Mapowanie koncept�w pomi�dzy schematami - SKOS dostarcza odpowiednich
      narz�dzi, pozwalaj�cych mapowa� mi�dzy sob� poszczeg�lne koncepty w 
      r�nych schematach:
      \begin{itemize}
        \item \textbf{skos:exactMatch} - dok�adne dopasowanie, koncepty
          s� identyczne
        \item \textbf{skos:closeMatch} - dopasowanie cz�ciowe, kt�re oznacza,
          �e koncepty mog� by� u�ywane zamiennie, jednak w ograniczonym
          zakresie
        \item \textbf{skos:broaderMatch} - dopasowanie konceptu o bardziej
          og�lnym znaczeniu
        \item \textbf{skos:narrowerMatch} - dopasowanie konceptu o bardziej
          szczeg�owym znaczeniu
        \item \textbf{skos:relatedMatch} - dopasowanie konceptu o powi�zanym
          znaczeniu z danym konceptem bez wyr�nienia hierarchii
      \end{itemize}
    \item Kolekcje - s�u�� do zgrupowania koncept�w, kt�re mog� zosta� wsp�lnie
      dzieli� t� sam� etykiet�:
      \begin{itemize}
        \item \textbf{skos:Collection} - kolekcja koncept�w
        \item \textbf{skos:OrderedCollection} - kolekcja koncept�w, kt�rych
          kolejno�� jest znacz�ca np w kolekcji ,,wykszta�cenie'' umie�ci�:
          \begin{enumerate}
            \item podstawowe
            \item �rednie
            \item wy�sze
          \end{enumerate}
        \item \textbf{skos:member} - przypisuje do danej kolekcji, wybrany
          koncept, u�ycie:
\begin{verbatim}
<kolekcja> skos:member <koncept>
\end{verbatim}
        \item \textbf{skos:memberList} - przypisuje do danej kolekcji, list�
          koncept�w, u�ycie:
\begin{verbatim}
<kolekcja> skos:memberList (<koncept1> <koncept2>)
\end{verbatim}
      \end{itemize}
\end{itemize}
\bibliography{BudowaModeliInformacyjnych}
\addcontentsline{toc}{section}{Bibliografia}

\end{document}

\end{document}



\end{document}